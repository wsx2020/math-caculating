%!TEX program = xelatex
\documentclass[UTF8]{ctexart}
\title{数学运算大赛}
\author{六盘水市三中数学社}
\date{}

\usepackage{geometry}
\geometry{margin = 1.4in}
\usepackage{xcolor}
\usepackage{amsmath, amssymb}
\usepackage{hologo}
\usepackage{graphicx}
\usepackage{enumitem}
\setlist{noitemsep}
\ifxetex\else
  \usepackage{CJKspace}
\fi
\usepackage{hyperref}
\hypersetup{colorlinks}
\usepackage{etoolbox}
\makeatletter
\patchcmd{\@maketitle}{\LARGE \@title \par}{\Huge \bfseries \@title \par}{}{}
\makeatother

\newcommand{\XeLaTeX}{\hologo{XeLaTeX}}
\newcommand{\pdfLaTeX}{\hologo{pdfLaTeX}}
\renewcommand{\CTeX}{\ensuremath{\mathbb{C}}\!\TeX}
\newcommand{\mdesc}[1]{{\sffamily\kaishu\color{red} #1}}
\newcommand{\myem}[1]{{\bfseries\color{red} #1}}
\newcommand{\myemph}[1]{{\sffamily\kaishu\color{red!80!orange} #1}}
\newcommand{\Warning}{\myem{\Large \rule{0pt}{1.4em}内部资料,请勿外传}}
\newcommand{\warning}{\cleaders\hbox to \linewidth{\hss\Warning\hss}\vfill}

\begin{document}
\maketitle
\warning
\clearpage

\section{运算大赛赛题第一轮}

\noindent 1. 已知方程组

\begin{equation*}
\left\{
\begin{aligned}
x^2+2y^2=8&,& \\
y=kx+4& &
\end{aligned}
\right.
\end{equation*}
中,$k$为常数,其解有$(x_1,y_1)$、$(x_2,y_2)$. 试化简$\Delta>0$和$y_1\cdot y_2$.
\vspace{3cm}


1答案:$\Delta>0$ $\Rightarrow$ $k^2>\dfrac32$\;($k<-\dfrac{\sqrt6}2$或$k>\dfrac{\sqrt6}2)$.
  $$y_1\cdot y_2=\dfrac{-40k^2}{2k^2+1}+16=\dfrac{16-8k^2}{2k^2+1}.$$


  \newpage

\noindent 2. 在方程组\begin{equation*}
\left\{
\begin{aligned}
 y=k(x-1)&,& \\
 3x^2+4y^2=12& &
\end{aligned}
\right.
\end{equation*}中,$k$为常数, $x_1$、$x_2$为方程组的两个根. \\
试求$x_1+x_2$ 和$x_1\cdot x_2$ .
 \vspace{3cm}

   
2答案: $$x_1+ x_2=\dfrac{8k^2}{4k^2+3}=2-\dfrac{6}{4k^2+3}.$$
 $$x_1\cdot x_2=\dfrac{4k^2-12}{4k^2+3}=1-\dfrac{15}{4k^2+3}.$$
  

 \newpage

\noindent 3. 在方程组\begin{equation*}
\left\{
\begin{aligned}
 y=kx+b&,& \\
 y^2=8x& &
\end{aligned}
\right.
\end{equation*}中$k$、$b$为常数, $x_1$、$x_2$为方程组的两个根. \\
(1)化简 $\Delta>0$;\\
(2)求$x_1+ x_2$和$x_1\cdot x_2$.\vspace{3cm}

%
3答案:$\Delta>0$ $\Rightarrow$ $kb<2$.\\
$$x_1+ x_2=\dfrac{8-2kb}{k^2}$$
 $$x_1\cdot x_2=\dfrac{b^2}{k^2}.$$


 \newpage

\noindent 4. 关于$x$、$y$的方程组\begin{equation*}
\left\{
\begin{aligned}
 \dfrac{x^2}4+\dfrac{y^2}m=n&,& \\
 y=x+m & &
\end{aligned}
\right.
\end{equation*}中,$m\neq4$、$m\neq n\neq 0$,$x_1$、$x_2$为方程组的两个根. 化简$\dfrac{x_1+ x_2}{x_1\cdot x_2}$ 的值.\vspace{3cm}

   
4答案: $$\dfrac{x_1+ x_2}{x_1\cdot x_2}=\dfrac{2}{n-m}.$$

\newpage

 \noindent 5. 直线$OB$方程为$y=-\dfrac1ax$,直线$BF$方程为$y=\dfrac1a(x-c)$.  \\
 求$B$ 点的坐标.\vspace{3cm}

   
5答案:$B(\dfrac{c}{2},-\dfrac{c}{2a}).$

\newpage
\section{运算大赛赛题第二轮}

\noindent 1. 关于$x$、$y$的方程组\begin{equation*}
\left\{
\begin{aligned}
 y=kx+m&,& \\
 \dfrac{x^2}{3}+\dfrac{y^2}2=1.& &
\end{aligned}
\right.
\end{equation*} $x_1$、$x_2$为其两个根. \\
 (1)化简$\Delta>0$;\\
 (2)求$x_1+x_2$ 和$x_1\cdot x_2$ 的值.
\vspace{3cm}

  1.答案:
  
  $\Delta>0$ $\Rightarrow$ $3k^2-m^2+2>0.$
    
  
  $$x_1+ x_2=\dfrac{-6km}{2+3k^2}.$$
  
  $$x_1\cdot x_2=\dfrac{3(m^2-2)}{2+3k^2}.$$


\newpage

 \noindent 2. 在方程组\begin{equation*}
\left\{
\begin{aligned}
 y=-2(x-\sqrt{5})&,& \\
 15x^2-32\sqrt5x+84=0& &
\end{aligned}
\right.
\end{equation*}中, $(x_1,y_1)$、$(x_2,y_2)$为其两组解. 试求$(x_1,y_1)$与$(x_2,y_2)$.

   
2答案: 

$$(\dfrac{6\sqrt5}{5},-\dfrac{2\sqrt5}{5}),$$
 
$$(\dfrac{14\sqrt5}{15},\dfrac{2\sqrt5}{15}).$$


\newpage

\noindent 3. 在方程组\begin{equation*}
\left\{
\begin{aligned}
 y=kx+m&,& \\
\dfrac{x^2}{4}+\dfrac{y^2}3=1 & &
\end{aligned}
\right.
\end{equation*}中, $x_1$、$x_2$为其两个根. \\
 (1)化简$\Delta>0$;\\
 (2)求$x_1+x_2$ 和$x_1\cdot x_2$ 的值.
\vspace{3cm}

  
  3答案:
  $\Delta>0$ $\Rightarrow$ $4k^2-m^2+3>0.$
    
  $$x_1+ x_2=\dfrac{-8km}{4k^2+3}.$$
  
  $$x_1\cdot x_2=\dfrac{4m^2-12}{4k^2+3}.$$

\newpage


 \noindent 4. 在方程组\begin{equation*}
\left\{
\begin{aligned}
x=my+4 &,& \\
\dfrac{x^2}{4}+\dfrac{y^2}3=1 & &
\end{aligned}
\right.
\end{equation*}中, $y_1$、$y_2$为其两个根. \\
 求$y_1+y_2$ 及$y_1\cdot y_2$ .
\vspace{3cm}

   
4答案:    
$$y_1+ y_2=\dfrac{-24m}{3m^2+4}.$$
  
$$y_1\cdot y_2=\dfrac{36}{3m^2+4}.$$

\newpage

\noindent 5. 在方程组\begin{equation*}
\left\{
\begin{aligned}
y=kx+m &,& \\
 \dfrac{x^2}4+\dfrac{y^2}3=1& &
\end{aligned}
\right.
\end{equation*}中$k$,$m$为常数.  当$\Delta=0$ 时,用$k$、$m$表示其交点的横坐标.
\vspace{3cm}

5.答案: $$(-\dfrac{4k}m,\dfrac3m).$$

  
\newpage
\section{积分赛} % (fold)


\noindent 1. 在方程组\begin{equation*}
    \left\{
    %%%%%%%%%%%%%%%%%%%%%%
    \begin{aligned}
    x=my+\dfrac{m^2}{2} &,& \\
     \dfrac{x^2}{m^2}+y^2=1& &
    \end{aligned}
    %%%%%%%%%%%%%%%%%%%%%%
    \right.
    \end{equation*}中,$m$为常数, $y_1$、$y_2$为其两个根. 试求$y_1$ 和$y_2$ 的值.\vspace{3cm}
    
       %%%%%%%%%%%%%%%%%%%%%%%%%%%%%%
\noindent 1.答案: $$y_1=\dfrac{-m-\sqrt{8-m^2}}{4},$$
     $$y_2=\dfrac{-m+\sqrt{8-m^2}}{4}.$$
      %%%%%%%%%%%%%%%%%%%%%%%%%%%%%
      \newpage
    
\noindent  2. 在方程组\begin{equation*}
    \left\{
    %%%%%%%%%%%%%%%%%%%%%%
    \begin{aligned}
     \dfrac{x^2}{25}+\dfrac{y^2}{16}=1&,& \\
     my=nkx+n& &
    \end{aligned}
    %%%%%%%%%%%%%%%%%%%%%%
    \right.
    \end{equation*}中,$m$、$n$均为常数且$m\neq0$, $(x_1,y_1)$、$(x_2,y_2)$为其两组解.  
    当$\dfrac{m}{n}=1$时,求$x_1$ 和$x_2$ 的值.\vspace{3cm}
    
       %%%%%%%%%%%%%%%%%%%%%%%%%%%%%%
{\color{red}
\noindent 2.答案:

$$x_1=\dfrac{-25k-5\sqrt{375k^2+265}}{25k^2+16},$$
     
$$x_2=\dfrac{-25k+5\sqrt{375k^2+265}}{25k^2+16}.$$
}
      %%%%%%%%%%%%%%%%%%%%%%%%%%%%%
      \newpage
    
\noindent 3. 在方程组\begin{equation*}
    \left\{
    %%%%%%%%%%%%%%%%%%%%%%
    \begin{aligned}
     x=my+\sqrt{3}&,& \\
     \dfrac{x^2}{6}+\dfrac{y^2}{3}=1& &
    \end{aligned}
    %%%%%%%%%%%%%%%%%%%%%%
    \right.
    \end{equation*}中,$m$为常数, $(x_1,y_1)$、$(x_2,y_2)$为其两组解. 
    试求$y_1$ 和$y_2$ 的值.\vspace{3cm}
    
       %%%%%%%%%%%%%%%%%%%%%%%%%%%%%%


\noindent 3.答案: 
{\color{red}
$$y_1=\dfrac{-\sqrt{3}m-\sqrt{6+6m^2}}{m^2+2},$$
     
$$y_2=\dfrac{-\sqrt{3}m+\sqrt{6+6m^2}}{m^2+2}.$$
}
      %%%%%%%%%%%%%%%%%%%%%%%%%%%%%
      \newpage
    
\noindent 4.在方程$$\dfrac{a}{m}x^3+9x^2+6mx+m^2=0$$中,
当$\dfrac{3a^2-9a}{\frac{a-3}{m}}=0$有意义时, 求$x$ 的值.\vspace{3cm}
    
       %%%%%%%%%%%%%%%%%%%%%%%%%%%%%%
\noindent 4.答案: $$x=-\dfrac{m}{3}\;(m\neq0).$$
      %%%%%%%%%%%%%%%%%%%%%%%%%%%%%
      \newpage
    
\noindent 5. 在关于$x$、$y$的方程组\begin{equation*}
    \left\{
    %%%%%%%%%%%%%%%%%%%%%%
    \begin{aligned}
     y^2=4x&,& \\
     y=-\frac{y_0}{2}x+b& &
    \end{aligned}
    %%%%%%%%%%%%%%%%%%%%%%
    \right.
    \end{equation*}中,当方程只有一组解时,求$b$与$y_0$的关系.\vspace{3cm}
    
       %%%%%%%%%%%%%%%%%%%%%%%%%%%%%%
\noindent 5.答案: $$b=-\dfrac{2}{y_0}.$$
      %%%%%%%%%%%%%%%%%%%%%%%%%%%%%
      \newpage
    
\noindent 6. 解方程组\begin{equation*}
    \left\{
    %%%%%%%%%%%%%%%%%%%%%%
    \begin{aligned}
     \dfrac{x}{c}+\dfrac{y}{b}=1&,& \\
     \dfrac{x^2}{a^2}+\dfrac{y^2}{b^2}=1&,&
    \end{aligned}
    %%%%%%%%%%%%%%%%%%%%%%
    \right.
    \end{equation*}其中,$a$,$b$,$c$为常数.\vspace{3cm}
    
       %%%%%%%%%%%%%%%%%%%%%%%%%%%%%%
\noindent 6.答案: $$x_1=\dfrac{2a^2c}{a^2+c^2}; y_1=\dfrac{b(c^2-a^2)}{a^2+c^2}.$$
     $$x_2=0; y_2=b.$$
      %%%%%%%%%%%%%%%%%%%%%%%%%%%%%
      \newpage
    
\noindent 7. 在关于$x$、$y$的方程组\begin{equation*}
    \left\{
    %%%%%%%%%%%%%%%%%%%%%%
    \begin{aligned}
     \dfrac{x^2}{a^2}+\frac{y^2}{b^2}=1&,& \\
     y=\frac12x-\frac12& &
    \end{aligned}
    %%%%%%%%%%%%%%%%%%%%%%
    \right.
    \end{equation*}中,$a\neq0$,$b\neq0$, $(x_1,y_1)$、$(x_2,y_2)$为其两组解. 求$y_1$ 和$y_2$.\vspace{3cm}
    
    
       %%%%%%%%%%%%%%%%%%%%%%%%%%%%%%
\noindent 7.答案: $$y_1=\dfrac{-2b^2-ab\sqrt{a^2+4b^2-1}}{a^2+4b^2},$$
     $$y_2=\dfrac{-2b^2+ab\sqrt{a^2+4b^2-1}}{a^2+4b^2}.$$
      %%%%%%%%%%%%%%%%%%%%%%%%%%%%%
      \newpage
    
\noindent 8. 在方程组\begin{equation*}
    \left\{
    %%%%%%%%%%%%%%%%%%%%%%
    \begin{aligned}
    \dfrac{x^2}{4}-\dfrac{y^2}{3}=1 &,& \\
     y=kx-5& &
    \end{aligned}
    %%%%%%%%%%%%%%%%%%%%%%
    \right.
    \end{equation*}中$k$为常数,有两组解$(x_1,y_1)$、$(x_2,y_2)$. 求$y_1$ 和$y_2$.\vspace{3cm}
    
       %%%%%%%%%%%%%%%%%%%%%%%%%%%%%%
\noindent 8.答案: $$y_1=\dfrac{15-4k\sqrt{21-3k^2}}{4k^2-3},$$
     $$y_2=\dfrac{15+4k\sqrt{21-3k^2}}{4k^2-3}.$$
      %%%%%%%%%%%%%%%%%%%%%%%%%%%%%
      \newpage
    
\noindent 9. 在方程组\begin{equation*}
    \left\{
    %%%%%%%%%%%%%%%%%%%%%%
    \begin{aligned}
    y^2=3px &,& \\
     y=x+m& &
    \end{aligned}
    %%%%%%%%%%%%%%%%%%%%%%
    \right.
    \end{equation*}中,$p$、$m$为常数,有两组解$(x_1,y_1)$、$(x_2,y_2)$. 试求$y_1$ 和$y_2$的值.\vspace{3cm}
    
       %%%%%%%%%%%%%%%%%%%%%%%%%%%%%%
\noindent 9.答案: $$y_1=\dfrac{3p-\sqrt{9p^2-12pm}}{2},$$
     $$y_2=\dfrac{3p+\sqrt{9p^2-12pm}}{2}.$$
      %%%%%%%%%%%%%%%%%%%%%%%%%%%%%
      %%%%%%%%%%%%%%%%%%%%%%%%%%%%%
      \newpage
    
\noindent 10. 在方程组\begin{equation*}
    \left\{
    %%%%%%%%%%%%%%%%%%%%%%
    \begin{aligned}
     x^2+3y^2=6&,& \\
     kx+2y-4=0& &
    \end{aligned}
    %%%%%%%%%%%%%%%%%%%%%%
    \right.
    \end{equation*}中,$k$为常数, $(x_1,y_1)$、$(x_2,y_2)$为其两组解. 求$y_1$ 和$y_2$的值.\vspace{3cm}
    
       %%%%%%%%%%%%%%%%%%%%%%%%%%%%%%
\noindent 10.答案: $$y_1=\dfrac{8-k\sqrt{18k^2-24}}{3k^2+4},$$
     $$y_2=\dfrac{8+k\sqrt{18k^2-24}}{3k^2+4}.$$
      %%%%%%%%%%%%%%%%%%%%%%%%%%%%%
      \newpage
    
\noindent 11. 在关于$x$、$y$的方程组\begin{equation*}
    \left\{
    %%%%%%%%%%%%%%%%%%%%%%
    \begin{aligned}
     mx^2+ny^2=2&,& \\
     y=-kx+1&,&
    \end{aligned}
    %%%%%%%%%%%%%%%%%%%%%%
    \right.
    \end{equation*}求$x_1$ 和$x_2$.\vspace{3cm}
    
     %%%%%%%%%%%%%%%%%%%%%%%%%%%%%%
\noindent 11.答案: $$x_1=\dfrac{-kn-\sqrt{2k^2n-mn+2m}}{k^2n+m},$$
     $$x_2=\dfrac{-kn+\sqrt{2k^2n-mn+2m}}{k^2n+m}.$$
      %%%%%%%%%%%%%%%%%%%%%%%%%%%%%
      \newpage
    
\noindent 12. 在方程组\begin{equation*}
    \left\{
    %%%%%%%%%%%%%%%%%%%%%%
    \begin{aligned}
     3x^2-y^2-3=0&,& \\
     y=-2x+m& &
    \end{aligned}
    %%%%%%%%%%%%%%%%%%%%%%
    \right.
    \end{equation*}有两组解$(x_1,y_1)$、$(x_2,y_2)$. 求$y_1^2$ 和$y_2^2$.\vspace{3cm}
    
       %%%%%%%%%%%%%%%%%%%%%%%%%%%%%%
\noindent 12.答案:$$y_1^2=21m^2-12-12m\sqrt{3m^2-3},$$
     $$y_2^2=21m^2-12+12m\sqrt{3m^2-3}.$$
      %%%%%%%%%%%%%%%%%%%%%%%%%%%%%
      \newpage
    
\noindent 13. 方程组\begin{equation*}
    \left\{
    %%%%%%%%%%%%%%%%%%%%%%
    \begin{aligned}
     x^2-\dfrac{y^2}{4}=1&,& \\
     kx-y+b=0 & &
    \end{aligned}
    %%%%%%%%%%%%%%%%%%%%%%
    \right.
    \end{equation*}中,$k$、$b$为常数, $(x_1,y_1)$、$(x_2,y_2)$为其两组解. 求$x_1$ 和$x_2$.\vspace{3cm}
    
      %%%%%%%%%%%%%%%%%%%%%%%%%%%%%%
\noindent 13.答案:$$y_1=\dfrac{kb-2\sqrt{b^2-k^2+4}}{4-k^2},$$
     $$y_2=\dfrac{kb+2\sqrt{b^2-k^2+4}}{4-k^2}.$$
      %%%%%%%%%%%%%%%%%%%%%%%%%%%%%
      \newpage
    
\noindent 14. 关于$x$、$y$的方程组\begin{equation*}
    \left\{
    %%%%%%%%%%%%%%%%%%%%%%
    \begin{aligned}
     y=kx+m&,& \\
     \dfrac{x^2}{a^2}+\frac{y^2}{b^2}=1 & &
    \end{aligned}
    %%%%%%%%%%%%%%%%%%%%%%
    \right.
    \end{equation*}{\color{red}有两组解 $(x_1,y_1)$、$(x_2,y_2)$,} 求$x_1$ 和$x_2$.\vspace{3cm}
    
     %%%%%%%%%%%%%%%%%%%%%%%%%%%%%%
\noindent 14.答案:$$x_1=\dfrac{-a^2km-a\sqrt{2a^2k^2m^2-a^2b^2k^2+b^2m^2+b^4}}{a^2k^2+b^2},$$
     $$x_2=\dfrac{-a^2km+a\sqrt{2a^2k^2m^2-a^2b^2k^2+b^2m^2+b^4}}{a^2k^2+b^2}.$$
      %%%%%%%%%%%%%%%%%%%%%%%%%%%%%
\newpage
    
\noindent 15. 关于$x$、$y$的方程组\begin{equation*}
    \left\{
    %%%%%%%%%%%%%%%%%%%%%%
    \begin{aligned}
    x=my+\sqrt{3} &,& \\
    \dfrac{x^2}{6}+\frac{y^2}{3}=1 & &
    \end{aligned}
    %%%%%%%%%%%%%%%%%%%%%%
    \right.
    \end{equation*}{\color{red}有两组解$(x_1,y_1)$、$(x_2,y_2)$,} 求$x_1$ 和$x_2$.\vspace{3cm}
    
      %%%%%%%%%%%%%%%%%%%%%%%%%%%%%%
\noindent 15.答案:$$x_1=\dfrac{2\sqrt3-m\sqrt{6m^2+6}}{m^2+2},$$
     $$x_2=\dfrac{2\sqrt3+m\sqrt{6m^2+6}}{m^2+2}.$$
      %%%%%%%%%%%%%%%%%%%%%%%%%%%%%

    


\end{document}
